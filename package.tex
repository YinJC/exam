
\usepackage{lastpage}
%\usepackage{times} %use the Times New Roman fonts
\usepackage{color}
%\usepackage{placeins}
\usepackage{ulem}
\usepackage{titlesec}
\usepackage{graphicx}
\usepackage{colortbl}
\usepackage{listings}
\usepackage{makecell}
\usepackage{indentfirst}
\usepackage{fancyhdr}
\usepackage{setspace} % 行间距
\usepackage{bm}%\boldsymbol 粗体
% 数学
\usepackage{amsmath,amsfonts,amssymb,times}
\usepackage[amsmath,thmmarks]{ntheorem}
\usepackage{txfonts}
\usepackage{tagging}
%\usetag{ans}% 注释掉该行语句不显示答案
\renewcommand*\CJKunderlinecolor{\color{black}}
\newcommand{\tkans}[1]{\iftagged{ans}{\underline{\ #1\ }.}{\underline{\ \phantom{#1}\ }.}}%
\newcommand{\xzans}[1]{\iftagged{ans}{(\ {#1}\ )}{(\ \phantom{#1}\ )}}%
\newcommand{\pdans}[1]{\iftagged{ans}{\hfill(\ {#1}\ )}{\hfill(\ \phantom{#1}\ )}}%
\usepackage{enumerate}% 编号
\usepackage{tikz,pgfplots} %绘图
\usepackage{tkz-euclide,pgfplots}
\usetikzlibrary{automata,positioning}
%\usepackage[paperwidth=18.4cm,paperheight=26cm,top=1.5cm,bottom=2cm,right=2cm]{geometry} % 单页
\usepackage[paperwidth=36.8cm,paperheight=26cm,top=2.5cm,bottom=2cm,right=2cm]{geometry}
\lstset{language=C,keywordstyle=\color{red},showstringspaces=false,rulesepcolor=\color{green}}
\oddsidemargin=0.5cm   %奇数页页边距
\evensidemargin=0.5cm %偶数页页边距
%\textwidth=14.5cm        %文本的宽度 单页
\textwidth=30cm        %文本的宽度 单页

\newsavebox{\zdx}%装订线

\newcommand{\putzdx}{\marginpar{
		\parbox{1cm}{\vspace{-1.6cm}
			\rotatebox[origin=c]{90}{
				\usebox{\zdx}
		}}
}}

\newcommand{\blank}{\uline{\textcolor{white}{a}\ \textcolor{white}{a}\ \textcolor{white}{a}\ \textcolor{white}{a}\ \textcolor{white}{a}\ \textcolor{white}{a}\ \textcolor{white}{a}\ \textcolor{white}{a}\ \textcolor{white}{a}\ \textcolor{white}{a}\ \textcolor{white}{a}}}

\newcommand{\me}{\mathrm{e}}  %定义 对数常数e,虚数符号i,j以及微分算子d为直立体。
\newcommand{\mi}{\mathrm{i}}
\newcommand{\mj}{\mathrm{j}}
\newcommand{\dif}{\mathrm{d}}
\newcommand{\bs}{\boldsymbol}%数学黑体
\newcommand{\ds}{\displaystyle}
%通常我们使用的分数线是系统自己定义的分数线,即分数线的长度的预设值是分子或分母所占的最大宽度,如何让分数线的长度变长成,我们%可以在分子分母添加间隔来实现。如中文分式的命令可以定义为:
%\newcommand{\chfrac[2]}{\cfrac{\;#1\;}{\;#2\;}}
%\frac{1}{2} \qquad \chfrac{1}{2}

\newcommand{\dfk}
{\begin{tabular}{|p{0.05\textwidth}|p{0.05\textwidth}|}
		\hline
		% after \\: \hline or \cline{col1-col2} \cline{col3-col4} ...
		\centering 阅卷人& \\
		\hline
		\centering 得~~分 &  \\
		\hline
\end{tabular}}
\newcounter{qst}
\renewcommand{\theqst}{\Chinese{qst}}
\newcommand{\qst}[1]{\dfk\,{\large\heiti\refstepcounter{qst}\theqst、#1}}

%选择题
%根据选项长短选择合适布局指令
\newcommand{\fourch}[4]{\\\begin{tabular}{*{4}{@{}p{3.5cm}}}(A)~#1; & (B)~#2; & (C)~#3; & (D)~#4.\end{tabular}} % 四行
\newcommand{\twoch}[4]{\\\begin{tabular}{*{2}{@{}p{7cm}}}(A)~#1; & (B)~#2;\end{tabular}\\\begin{tabular}{*{2}{@{}p{7cm}}}(C)~#3; &
		(D)~#4.\end{tabular}}  %两行
\newcommand{\onech}[4]{\\(A)~#1; \\ (B)~#2; \\ (C)~#3; \\ (D)~#4.}  % 一行

%根据选项长短自动选择布局
\newlength{\la}
\newlength{\lb}
\newlength{\lc}
\newlength{\ld}
\newlength{\lhalf}
\newlength{\lquarter}
\newlength{\lmax}
\newcommand{\xx}[4]{\ \\[.5pt]%
	\settowidth{\la}{(A)~#1;~~~}
	\settowidth{\lb}{(B)~#2;~~~}
	\settowidth{\lc}{(C)~#3;~~~}
	\settowidth{\ld}{(D)~#4.~~~}
	\ifthenelse{\lengthtest{\la > \lb}}
	{\setlength{\lmax}{\la}}
	{\setlength{\lmax}{\lb}}
	\ifthenelse{\lengthtest{\lmax < \lc}}
	{\setlength{\lmax}{\lc}}  {}
	\ifthenelse{\lengthtest{\lmax < \ld}}
	{\setlength{\lmax}{\ld}}  {}
	\setlength{\lhalf}{0.5\linewidth}
	\setlength{\lquarter}{0.25\linewidth}
	\ifthenelse{\lengthtest{\lmax < \lquarter}}
	{\noindent\makebox[\lquarter][l]{(A)~#1;~~~}%
		\makebox[\lquarter][l]{(B)~#2;~~~}%
		\makebox[\lquarter][l]{(C)~#3;~~~}%
		\makebox[\lquarter][l]{(D)~#4.~~~}}%
	{\ifthenelse{\lengthtest{\lmax < \lhalf}}
		{\noindent\makebox[\lhalf][l]{(A)~#1;~~~}%
			\makebox[\lhalf][l]{(B)~#2;~~~}\\%
			\makebox[\lhalf][l]{(C)~#3;~~~}%
			\makebox[\lhalf][l]{(D)~#4.~~~}}
		{\noindent{}(A)~#1;  \\ (B)~#2; \\ (C)~#3; \\ (D)~#4. }
	}
}

\theoremstyle{nonumberplain}
%\theoremsymbol{\ensuremath{\Box}}
\theorembodyfont{\rmfamily}
\theoremheaderfont{\sffamily}
\newtheorem{jieda}{解:}
\newtheorem{proof}{证明:}

\newcommand{\jd}[1]{\iftagged{ans}{\begin{jieda}
			#1
	\end{jieda}}{\ \\[5cm]}}

\renewcommand{\headrulewidth}{0pt}
\pagestyle{fancy}