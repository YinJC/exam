% !Mode::"TeX:UTF-8"
\documentclass[twocolumn,landscape,fntef,UTF8]{ctexart}

\usepackage{lastpage}
%\usepackage{times} %use the Times New Roman fonts
\usepackage{color}
%\usepackage{placeins}
\usepackage{ulem}
\usepackage{titlesec}
\usepackage{graphicx}
\usepackage{colortbl}
\usepackage{listings}
\usepackage{makecell}
\usepackage{indentfirst}
\usepackage{fancyhdr}
\usepackage{setspace} % 行间距
\usepackage{bm}%\boldsymbol 粗体
% 数学
\usepackage{amsmath,amsfonts,amssymb,times}
\usepackage[amsmath,thmmarks]{ntheorem}
\usepackage{txfonts}
\usepackage{tagging}
%\usetag{ans}% 注释掉该行语句不显示答案
\renewcommand*\CJKunderlinecolor{\color{black}}
\newcommand{\tkans}[1]{\iftagged{ans}{\underline{\ #1\ }.}{\underline{\ \phantom{#1}\ }.}}%
\newcommand{\xzans}[1]{\iftagged{ans}{(\ {#1}\ )}{(\ \phantom{#1}\ )}}%
\newcommand{\pdans}[1]{\iftagged{ans}{\hfill(\ {#1}\ )}{\hfill(\ \phantom{#1}\ )}}%
\usepackage{enumerate}% 编号
\usepackage{tikz,pgfplots} %绘图
\usepackage{tkz-euclide,pgfplots}
\usetikzlibrary{automata,positioning}
%\usepackage[paperwidth=18.4cm,paperheight=26cm,top=1.5cm,bottom=2cm,right=2cm]{geometry} % 单页
\usepackage[paperwidth=36.8cm,paperheight=26cm,top=2.5cm,bottom=2cm,right=2cm]{geometry}
\lstset{language=C,keywordstyle=\color{red},showstringspaces=false,rulesepcolor=\color{green}}
\oddsidemargin=0.5cm   %奇数页页边距
\evensidemargin=0.5cm %偶数页页边距
%\textwidth=14.5cm        %文本的宽度 单页
\textwidth=30cm        %文本的宽度 单页

\newsavebox{\zdx}%装订线

\newcommand{\putzdx}{\marginpar{
		\parbox{1cm}{\vspace{-1.6cm}
			\rotatebox[origin=c]{90}{
				\usebox{\zdx}
		}}
}}

\newcommand{\blank}{\uline{\textcolor{white}{a}\ \textcolor{white}{a}\ \textcolor{white}{a}\ \textcolor{white}{a}\ \textcolor{white}{a}\ \textcolor{white}{a}\ \textcolor{white}{a}\ \textcolor{white}{a}\ \textcolor{white}{a}\ \textcolor{white}{a}\ \textcolor{white}{a}}}

\newcommand{\me}{\mathrm{e}}  %定义 对数常数e,虚数符号i,j以及微分算子d为直立体。
\newcommand{\mi}{\mathrm{i}}
\newcommand{\mj}{\mathrm{j}}
\newcommand{\dif}{\mathrm{d}}
\newcommand{\bs}{\boldsymbol}%数学黑体
\newcommand{\ds}{\displaystyle}
%通常我们使用的分数线是系统自己定义的分数线,即分数线的长度的预设值是分子或分母所占的最大宽度,如何让分数线的长度变长成,我们%可以在分子分母添加间隔来实现。如中文分式的命令可以定义为:
%\newcommand{\chfrac[2]}{\cfrac{\;#1\;}{\;#2\;}}
%\frac{1}{2} \qquad \chfrac{1}{2}

\newcommand{\dfk}
{\begin{tabular}{|p{0.05\textwidth}|p{0.05\textwidth}|}
		\hline
		% after \\: \hline or \cline{col1-col2} \cline{col3-col4} ...
		\centering 阅卷人& \\
		\hline
		\centering 得~~分 &  \\
		\hline
\end{tabular}}
\newcounter{qst}
\renewcommand{\theqst}{\Chinese{qst}}
\newcommand{\qst}[1]{\dfk\,{\large\heiti\refstepcounter{qst}\theqst、#1}}

%选择题
%根据选项长短选择合适布局指令
\newcommand{\fourch}[4]{\\\begin{tabular}{*{4}{@{}p{3.5cm}}}(A)~#1; & (B)~#2; & (C)~#3; & (D)~#4.\end{tabular}} % 四行
\newcommand{\twoch}[4]{\\\begin{tabular}{*{2}{@{}p{7cm}}}(A)~#1; & (B)~#2;\end{tabular}\\\begin{tabular}{*{2}{@{}p{7cm}}}(C)~#3; &
		(D)~#4.\end{tabular}}  %两行
\newcommand{\onech}[4]{\\(A)~#1; \\ (B)~#2; \\ (C)~#3; \\ (D)~#4.}  % 一行

%根据选项长短自动选择布局
\newlength{\la}
\newlength{\lb}
\newlength{\lc}
\newlength{\ld}
\newlength{\lhalf}
\newlength{\lquarter}
\newlength{\lmax}
\newcommand{\xx}[4]{\ \\[.5pt]%
	\settowidth{\la}{(A)~#1;~~~}
	\settowidth{\lb}{(B)~#2;~~~}
	\settowidth{\lc}{(C)~#3;~~~}
	\settowidth{\ld}{(D)~#4.~~~}
	\ifthenelse{\lengthtest{\la > \lb}}
	{\setlength{\lmax}{\la}}
	{\setlength{\lmax}{\lb}}
	\ifthenelse{\lengthtest{\lmax < \lc}}
	{\setlength{\lmax}{\lc}}  {}
	\ifthenelse{\lengthtest{\lmax < \ld}}
	{\setlength{\lmax}{\ld}}  {}
	\setlength{\lhalf}{0.5\linewidth}
	\setlength{\lquarter}{0.25\linewidth}
	\ifthenelse{\lengthtest{\lmax < \lquarter}}
	{\noindent\makebox[\lquarter][l]{(A)~#1;~~~}%
		\makebox[\lquarter][l]{(B)~#2;~~~}%
		\makebox[\lquarter][l]{(C)~#3;~~~}%
		\makebox[\lquarter][l]{(D)~#4.~~~}}%
	{\ifthenelse{\lengthtest{\lmax < \lhalf}}
		{\noindent\makebox[\lhalf][l]{(A)~#1;~~~}%
			\makebox[\lhalf][l]{(B)~#2;~~~}\\%
			\makebox[\lhalf][l]{(C)~#3;~~~}%
			\makebox[\lhalf][l]{(D)~#4.~~~}}
		{\noindent{}(A)~#1;  \\ (B)~#2; \\ (C)~#3; \\ (D)~#4. }
	}
}



\renewcommand{\headrulewidth}{0pt}
\pagestyle{fancy}
\usetag{ans}% 注释掉该行语句不显示答案
\begin{document}
\newcommand{\school}{无名氏大学}
\def\kecheng{不知道写什么}
\def\xuenian{2017-2018}
\def\xueqi{1}
\fancyhf{}
\fancyfoot[CO,CE]{\vspace*{1mm}第\,\thepage\,页 , 共 ~\pageref{LastPage} 页}
\sbox{\zdx}
{\parbox{27cm}{\centering
	座位号~\underline{\makebox[34mm][c]{}}~ 班~级\underline{\makebox[34mm][c]{}}~\CJKfamily{song} 学~号\underline{\makebox[44mm][c]{}}~\CJKfamily{song} 姓~名\underline{\makebox[34mm][c]{}} ~\\
	\vspace{3mm}
请在所附答题纸上空出密封位置。并填写试卷序号、班级、学号和 姓名\\
%答题时学号
\vspace{1mm}
\dotfill{} 密\dotfill{}封\dotfill{}线\dotfill{} \\
	}}
\reversemarginpar
	
\begin{spacing}{1.25}
	\begin{center}
\begin{LARGE}
\school~\underline{~\xuenian~}\,学年第\,\underline{~\xueqi~}\,学期\\
\underline{~\kecheng~}\,\iftagged{ans}{试卷~参考答案和评分标准}{试卷}\\
\end{LARGE}
(闭卷笔试\ \ 90 分钟)\\
	\vspace{0.5cm}
\begin{tabular}{|m{0.03\textwidth}|*{8}{m{0.035\textwidth}|}p{0.04\textwidth}|}
	\hline
\centering  题~号 & \centering 一 & \centering 二 & \centering 三 & \centering 四& \centering 五 & \centering 六 & \centering 七 % & \centering 八 & \centering 九 & \centering 十
& \centering 总~分 & \makecell{阅卷\\教师} \rule{0pt}{3mm} \\
	\hline
	\centering 分~数 &  &  &  &  &  &  &  &  &  %&  &
	\rule{0pt}{8mm} \\\hline
				% \centering 计 &  &  &  &  &  &  &  &  &  &  & \\
				% \centering 分 &  &  &  &  &  &  &  &  &  &  & \\
				% \centering 人 &  &  &  &  &  &  &  &  &  &  & \\  \hline
\end{tabular}
\end{center}
\end{spacing}
\vspace{-0.5cm}
\setlength{\marginparsep}{1.7cm}
\putzdx %%装订线--奇页数
\vspace{1cm}


\qst{选择题~(每题~3 分,共~21 分)}
		
\begin{enumerate}\setcounter{enumi}{0}
\item 极限~$\lim\limits_{x\rightarrow \infty}\dfrac{\,\sin x\,}{x} = $\xzans{C}
\fourch{0}{1}{2}{$\infty$}
	
\item 如图,正方体$AC_1$的棱长为1,过点$A$作平面$A_1BD$的垂线,垂足为点$H$,则以下命题中,错误的命题是\xzans{D}

{\centering
		\begin{tikzpicture}[scale=1]
		\begin{scriptsize}
		\newcommand{\angleone}{25}
		\coordinate  [label=below:$A$] (A) at ({-1*cos(\angleone)},{-1*sin(\angleone)});
		\coordinate  [label=below:$B$] (B) at ({-1*cos(\angleone)+2},{-1*sin(\angleone)});
		\coordinate  [label=right:$C$] (C) at (2,0);
		\coordinate  [label=left:$D$] (D) at (0,0);
		\coordinate  [label=left:$A_1$] (A1) at ({-1*cos(\angleone)},{-1*sin(\angleone)+2});
		\coordinate  [label=right:$B_1$] (B1) at ({-1*cos(\angleone)+2},{-1*sin(\angleone)+2});
		\coordinate  [label=right:$C_1$] (C1) at (2,2);
		\coordinate  [label=left:$D_1$] (D1) at (0,2);
		\draw [densely dotted] (D)--(C) (D)--(D1) (D)--(A);
		\draw  [densely dotted] (A1)--(D) (C)--(D1) (B)--(D);
		\draw  (A1)--(B) (A)--(B) (C)--(C1) (C1)--(B1) (B1)--(B) (A)--(A1) (A1)--(B1) (B1)--(C1) (C1)--(D1) (A1)--(D1) (B)--(C) (B1)--(D1) (B1)--(C);
		\coordinate  (M) at ($(D)!0.5!(B)$) {};
		\coordinate  [label=above:$H$] (H) at ($(A1)!0.66!(M)$) {};
		\draw[densely dotted] (A)--(H);
		\end{scriptsize}
		\end{tikzpicture} \\
}
\xx{点$H$是$\triangle A_1BD$的垂心}{$AH\bot$平面$CB_1D_1$}{$AH$的延长线经过点$C_1$}{$AH$和$BB_1$所成角为$45^\circ$}
\item 下列说法正确的是\xzans{D}
\xx{分段函数一定不是初等函数}{若 $\lim\limits_{n\rightarrow \infty}x_ny_n=0$, 则必有 $\lim\limits_{n\rightarrow \infty}x_n=0$ 或 $\lim\limits_{n\rightarrow \infty}y_n=0$}{若 $f(x)$ 在 $(a,b) $ 内连续,则$f(x)$ 在 $(a,b) $ 内必有界}{若$\lim\limits_{n\rightarrow \infty}x_n=a$($a$ 为有限实数),则数列 $\{x_n\}$ 必有界}
			
\item 方程~$4x^2+y^2+z^2=4$ 表示的曲面方程是\xzans{C}
\xx{单叶双曲面.}{双叶双曲面.}{椭球面.}{抛物面.}
			
\item 二元函数~$f(x,y)$ 在点~$(x_0,y_0)$ 处两个偏导数~$f_x(x_0,y_0),f_y(x_0,y_0)$ 存在是~$f(x,y)$ 在该点连续的\xzans{D}
\twoch{充分而非必要条件.}{必要而非充分条件.}{充分必要条件.}{既非充分也非必要条件.}

\item 设有平面区域~$D=\{(x,y)\mid -a\leqslant x\leqslant a, x\leqslant y\leqslant a\}$ , ~$D_1=\{(x,y)\mid 0\leqslant x\leqslant a, x\leqslant y\leqslant a\}$ , 则~$\displaystyle{\iint\limits_{D}(xy+\cos x\sin y)\dif x\dif y}=$\xzans{D}
\xx{~$0$.}{~$4\displaystyle{\iint\limits_{D_1}(xy+\cos x\sin y)\dif x\dif y}$.}{~$2\displaystyle{\iint\limits_{D_1}xy\dif x\dif y}$.}{~$2\displaystyle{\iint\limits_{D_1}\cos x\sin y\dif x\dif y}$.}

\item 设~$L$ 为正向单位圆周~$x^2+y^2=1$, 则~$\ds{\oint_L(2xy-y)\dif x + (x^2 +2 x)\dif y} = $\xzans{A}
\xx{$3\pi$}{$2\pi$}{$\pi$}{$1$}
\end{enumerate}

\qst{判断题:~正确~$\surd$, 错误~$\times$ (每题~2 分,共~10分)}

\begin{enumerate}%\setcounter{enumi}{5}
			
\item 若~$f(x)$ 在~$(a,b)$ 上连续,则~$f(x)$ 在~$(a,b)$ 上一定可导. \pdans{$\times$}%\hfill$(~~~~\times~~~~)$
\item 函数 $f(x)$ 在 $x=x_0$ 处可导是函数 $f(x)$ 在 $x=x_0$ 处可微的充要条件.\pdans{$\surd$}% \hfill $(~~~~\surd~~~~)$
\item 函数 $f(x)=x^5+x-1$ 在 $(0,1)$ 内存在唯一解.\pdans{$\surd$}% \hfill $(~~~~\surd~~~~)$
\item $M(0,0)$ 为~$f(x,y) = x^6 + \sin^2(x\,y)$ 的一个极小值点.\pdans{$\surd$}% \hfill $(~~~~\surd~~~~)$
\item 若 ~$\sum\limits_{n=1}^\infty u_n$ 与~$\sum\limits_{n=1}^\infty v_n$ 都发散, 则~$\sum\limits_{n=1}^\infty (u_n+v_n)$ 也一定发散.\pdans{$\times$}%\hfill$(~~~~\times~~~~)$			
\end{enumerate}

\qst{填空题~(每题~3 分, 共~ 15 分)}
\begin{enumerate}%\setcounter{enumi}{10}
\item ~$\lim\limits_{x\rightarrow \infty}(1-x)^{\frac{\,1\,}{x}}=$\tkans{$\me^{-1}$}% ~\underline{~~~$\me^{-1}$~~~}.
			
\item 设~$z = u^2\ln v$ , 而~$u= \dfrac{x}{y}, v = x-y$ , 则~$\dfrac{\partial z}{\partial x}$=\tkans{$\dfrac{2x}{y^2}\ln(x-y)+\dfrac{x^2}{y^2(x-y)}$}% ~\underline{~~$\dfrac{2x}{y^2}\ln(x-y)+\dfrac{x^2}{y^2(x-y)}$~~}.
			
\item 函数~$f(x,y) = x\me^y$ 在点~$(1,0)$ 处的梯度为~$\nabla f = $\tkans{$(1,2)$}
			
\item 把二次积分~$\displaystyle{\int_0^1 \dif x \int_0^{\sqrt{1-x^2}} f(x,y) \dif y}$ 化为极坐标形式的二次积分为\\
\tkans{$\displaystyle{\int_0^{\pi/2} \dif \theta \int_0^1 f(\rho \cos\theta, \rho\sin\theta)\rho \dif \rho}$}

\item 设幂级数~$\sum\limits_{n=0}^\infty a_nx^n$ 的收敛半径为~$3$, 则幂级~$\sum\limits_{n=0}^\infty a_nx^{2n}$ 的收敛半径为\tkans{$\sqrt{3}$}
\end{enumerate}
		
\newpage
\putzdx %%装订线--奇页数
		
\qst{多元函数微分法~(每题~7分, 共~21分)}
\begin{enumerate}%\setcounter{enumi}{15}
\item 设~$\bs{a}=(3,4,5),~\bs{b}=(1,-2,3)$, 求~$\bs{a}\cdot\bs{b}$, $\bs{a}$ 在 $ \bs{b}$ 上的投影, ~$\bs{a}\times\bs{b}$.
\jd{~$\bs{a}\cdot\bs{b} = 3-8+15=10$ \dotfill{}(2')
		
		~$(\bs{a})_{\bs{b}} = \dfrac{\bs{a}\cdot\bs{b}}{|\bs{b}|}=\dfrac{10}{\sqrt{14}}$ \dotfill{}(2')
		
		~$\bs{a}\times\bs{b} = \begin{vmatrix}
		\bs{i} & \bs{j} & \bs{k}\\
		3 & 4 & 5\\
		1 & -2 & 3\\
		\end{vmatrix}=(22,-4,-10).$ \dotfill{}(3')}

	
\item  求过点~$A(1,2,-1), B(2,3,0),C(3,3,2)$ 的三角形~$\triangle ABC$ 的面积和它们确定的平面方程.

解:由题设~$\overrightarrow{AB}=(1,1,1),\overrightarrow{AC}=(2,1,3)$, \dotfill{}(2')

故~$\overrightarrow{AB}\times \overrightarrow{AC}=\begin{vmatrix}
\bs{i}&\bs{j} &\bs{k}\\
1&1&1\\
2&1&3\\
\end{vmatrix}=(2,-1,-1)$,

三角形~$\triangle ABC$ 的面积为~$S_{\triangle ABC}=\dfrac{1}{2}\mid\overrightarrow{AB}\times \overrightarrow{AC}\mid=\dfrac{1}{2}\sqrt{6}.$  \dotfill{}(2')

所求平面的方程为~$2(x-2)-(y-3)-z=0$, 即~$2x-y-z-1=0$
 \dotfill{}(3')			

\item 设函数~$z = f(u,v)$ 具有一阶连续偏导数,~$z = f(x^2+y^2,\dfrac{x}{y})$, 求~$\dfrac{\partial z}{\partial x}, \dfrac{\partial z}{\partial y}$, 并写出全微分~$\dif z$.
			
解: $\dfrac{\partial z}{\partial x} = 2xf_1' + \dfrac{1}{y}f_2'$, \dotfill{}(3')
			
$\dfrac{\partial z}{\partial y} = 2y f_1' - \dfrac{x}{y^2}f_2'$, \dotfill{}(3')
			
$\dif z =\dfrac{\partial z}{\partial x}\dif x + \dfrac{\partial z}{\partial y} \dif y\\	=(2xf_1' + \dfrac{1}{y}f_2')\dif x +(2y f_1' - \dfrac{x}{y^2}f_2')\dif y$, \dotfill{}(1')
\end{enumerate}
\vspace{2cm}

\newpage
\section*{\hspace{5cm}五、重积分~(每题~7 分, 共~ 21 分)}
\vspace{-1cm}
\begin{tabular}{|p{0.05\textwidth}|p{0.05\textwidth}|}
\hline
			% after \\: \hline or \cline{col1-col2} \cline{col3-col4} ...
\centering 阅卷人& \\
\hline
\centering 得~~分 &  \\
\hline
\end{tabular}
		
		\begin{enumerate}\setcounter{enumi}{18}
			\item 计算二重积分~$\ds{\iint\limits_{D}\dfrac{\sin x}{x}\dif \sigma}$, 其中~~$D=\{(x,y)\mid 0\leqslant x \leqslant \pi, 0\leqslant y \leqslant \pi x\}$.
			
			解:
			~$\ds{\iint\limits_{D}\dfrac{\sin x}{x}\dif \sigma}= \ds{\int_0^\pi\dfrac{\sin x}{x}\dif x\int_{0}^{\pi x}\dif y}$  \dotfill{}(2')
			
			~$= \ds{\int_0^\pi\dfrac{\sin x}{x}(\pi x- 0)\dif x = \pi\int_0^\pi\sin x\dif x  }$  \dotfill{}(3')
			
			~$=  \pi\Big[-\cos x \Big]_0^\pi = 2\pi $  \dotfill{}(2')
			
			\item 计算二重积分~$\ds{\iint\limits_{D}\me^{x^2+y^2}\dif \sigma}$, 其中~$D=\{(x,y)\Big| x^2+ y^2 \leqslant 25\}$.
\begin{tikzpicture}[domain=0:4]
\tkzInit[xmax=4.2,ymax=3.2,xmin=-1.2,ymin=-1.2,xstep=1]
%\tkzGrid
\tkzAxeXY
\draw[color=red] plot (\x,\x) node[right] {$f(x)=x$};
\draw[color=orange,domain=-0.5:4] plot (\x,{0.05*exp(\x)}) node[right] {$f(x)=\frac{1}{20}\mathrm e^x$};
\draw[color=blue,domain=0:4] plot (\x,{sin(\x r)}) node[right] {$f(x)=\sin x$};
\draw[color=blue!50,x=1cm,y=0.5cm,domain=-0.5:2.4] plot (\x, {(\x)^3-4*(\x)+2}) node[right] {$f(x)=x^3-4x+2$};
\end{tikzpicture}

			解:~$D: 0\leqslant \theta \leqslant 2\pi, 0\leqslant \rho \leqslant 5$  \dotfill{}(2')
			
			~$\ds{\iint\limits_{D}\me^{x^2+y^2}\dif \sigma}= \ds{\int_0^{2\pi}\dif \theta\int_0^5\me^{\rho^2}\rho\dif \rho}$  \dotfill{}(2')
			
			~$= \ds{2\pi \cdot\Big[\frac{1}{2}\me^{\rho^2}\Big]_0^5 = \pi(\me^{25}-1)} $  \dotfill{}(3')
			
			\vspace{1cm}
			\item 计算三重积分~$\ds{\iiint\limits_{\Omega}z \dif x\dif y\dif z}$, 其中~$\Omega$ 是由~圆锥面~$z = \sqrt{x^2+ y^2}$ 和平面~$z=4$ 围成的闭区域.
			
			解:~$\Omega = \{(x,y,z)\mid x^2 + y^2 \leqslant z^2,0\leqslant z\leqslant 4\}$ \dotfill{}(2')
			
			~$ \ds{\iiint\limits_{\Omega}z \dif x\dif y\dif z =  \int_0^4 z \dif z \iint\limits_{D_z}\dif x\dif y}$ \dotfill{}(3')
			
			~$  = \ds{\int_0^4z\times \pi z^2\dif z= \dfrac{\pi z^4}{4}\Big|_0^4=64\pi}$ \dotfill{}(2')
		\end{enumerate}
		\newpage
		\putzdx %%装订线--奇数页
		\section*{\hspace{5cm} 六、无穷级数~(本题~13 分)}
		\vspace{-2cm}
		\begin{tabular}{|p{0.05\textwidth}|p{0.05\textwidth}|}
			\hline
			% after \\: \hline or \cline{col1-col2} \cline{col3-col4} ...
			\centering  阅卷人& \\
			\hline
			\centering 得~~分 &  \\
			\hline
		\end{tabular}

		\begin{enumerate}\setcounter{enumi}{21}
			\item 求幂级数~$\ds{\sum\limits_{n=1}^\infty\dfrac{(x-1)^n}{n\cdot3^n}} $ 的收敛域与和函数~$s(x)$.
			
			解:令~$t = x-1$, 上述级数变为$\ds{\sum\limits_{n=1}^\infty\dfrac{(x-1)^n}{n\cdot3^n}} $  \dotfill{}(2')
			
			因为 $\rho = \lim\limits_{n\rightarrow\infty}\Big| \dfrac{a_{n+1}}{a_{n}}\Big| = \lim\limits_{n\rightarrow\infty}\dfrac{n\cdot3^n}{(n+1)\cdot3^{n+1}} = \dfrac{1}{3}$, \dotfill{}(2')
			
			所以,收敛半径$R=3 $ 收敛区间~$|t|<3$, 即~$-2<x<4$. \dotfill{}(3')
			
			当~$x = 4$ 时,级数变为~$\sum\limits_{n=1}^\infty\dfrac{1}{n}$ , 这级数发散, 当~$x = -2$ 时,级数变为~$\sum\limits_{n=1}^\infty\dfrac{(-1)^n}{n}$ , 这级数收敛, 原级数的收敛域为~$[-2,4)$.  \dotfill{}(2')
			
			设~$s(x) = \sum\limits_{n=1}^\infty\dfrac{(x-1)^n}{n\cdot3^n}$ , 则
			
			$s'(x) = \sum\limits_{n=1}^\infty\left(\dfrac{(x-1)^n}{n\cdot3^n}\right)' = \sum\limits_{n=1}^\infty\dfrac{(x-1)^{n-1}}{3^n} = \dfrac{\frac{1}{3}}{1-\frac{x-1}{3}}=\dfrac{1}{4-x}$
			
			$s(1) = 0$, $s(x) = s(1)+\ds{\int_1^x s'(t)\dif t = 0 + \int_1^x\dfrac{1}{4-t}\dif t }$
			
			$\ds{ =-\ln(4-t)\Big|_1^x = \ln3-\ln(4-x),-2\leqslant x<4.}$ \dotfill{}(4')
		\end{enumerate}
	
	\begin{tikzpicture}
	\begin{axis}[x=.5cm,xmin=0,ymin=0]
	\addplot[mark=none,smooth,red,thick] expression[domain=0:12]{exp(((x-6)^2)/(-9))};
	\addplot[mark=none,smooth,blue,thick] expression[domain=1:19]{exp(((x-10)^2)/(-25))};
	\addplot[mark=none,smooth,ultra thick] expression[domain=7.5:12]{exp(((x-6)^2)/(-9))};
	\addplot[mark=none,smooth,ultra thick] expression[domain=1:7.5]{exp(((x-10)^2)/(-25))};
	\addplot[dotted,mark=none]coordinates{(6,0)(6,1)};
	\addplot[dotted,mark=none]coordinates{(10,0)(10,1)(0,1)};
	\addplot[dashed,mark=none]coordinates{(7.5,0)(7.5,0.7788)(0,0.7788)};
	\node[pin=-45:{$P$}] at (axis cs:0,0.7788) {};
	\node[pin=135:{$P_x$}] at (axis cs:7.5,0) {};
	\end{axis}
	\end{tikzpicture}
%	\end{spacing}
\newpage
\begin{tikzpicture}
\begin{axis}
\addplot[color=red]{exp(x)};
\end{axis}
\end{tikzpicture}
%Here ends the furst plot
\hskip 5pt
%Here begins the 3d plot
\begin{tikzpicture}
\begin{axis}
\addplot3[
surf,
]
{exp(-x^2-y^2)*x};
\end{axis}
\end{tikzpicture}
文字环绕 文字环绕文字环绕文字环绕文字环绕文字环绕文字环绕文字环绕
\begin{tikzpicture}
\begin{axis}
\addplot3[
surf,
]
{exp(-x^2-y^2)*x};
\end{axis}
\end{tikzpicture}

	\clearpage
	
\end{document}
